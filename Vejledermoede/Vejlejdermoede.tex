\documentclass[12pt]{article}
\usepackage[english]{babel}
\usepackage[utf8x]{inputenc}
\usepackage{amsmath}
\usepackage{graphicx}
\usepackage[colorinlistoftodos]{todonotes}

\begin{document}

\begin{titlepage}

\newcommand{\HRule}{\rule{\linewidth}{0.5mm}} % Defines a new command for the horizontal lines, change thickness here

\center % Center everything on the page
 
%----------------------------------------------------------------------------------------
%	HEADING SECTIONS
%----------------------------------------------------------------------------------------

\textsc{\LARGE Aarhus Universitet}\\[1.5cm] % Name of your university/college
\textsc{\Large Semesterprojekt 3}\\[0.5cm] % Major heading such as course name
\textsc{\large Gruppe 7}\\[0.5cm] % Minor heading such as course title

%----------------------------------------------------------------------------------------
%	TITLE SECTION
%----------------------------------------------------------------------------------------

\HRule \\[0.4cm]
{ \huge \bfseries Vejledermøder}\\[0.4cm] % Title of your document
\HRule \\[1.5cm]
 
%----------------------------------------------------------------------------------------
%	AUTHOR SECTION
%----------------------------------------------------------------------------------------

\begin{minipage}{0.4\textwidth}
\begin{flushleft} \large
\emph{Forfatter:}\\
Tristan \textsc{Møller} % Your name
\end{flushleft}
\end{minipage}
~
\begin{minipage}{0.4\textwidth}
\begin{flushright} \large
\emph{Vejleder:} \\
Martin Ansbjerg \textsc{Kjær} % Supervisor's Name
\end{flushright}
\end{minipage}\\[2cm]

% If you don't want a supervisor, uncomment the two lines below and remove the section above
%\Large \emph{Author:}\\
%John \textsc{Smith}\\[3cm] % Your name

%----------------------------------------------------------------------------------------
%	DATE SECTION
%----------------------------------------------------------------------------------------

{\large \today}\\[2cm] % Date, change the \today to a set date if you want to be precise

%----------------------------------------------------------------------------------------
%	LOGO SECTION
%----------------------------------------------------------------------------------------


%----------------------------------------------------------------------------------------

\vfill % Fill the rest of the page with whitespace

\end{titlepage}

\section{Vejledermøde den 17-09-18}
\subsection{Mødeindkaldelse}
Dato: 17-09-2018
\\Tid: 13.00
\\Sted: Kælder (Under bibliotek)
\\Deltagere: Gruppe 7 og vejleder
\\Dagsorden:
\begin{enumerate}
    \item Præsentation af gruppemedlemmer og vejleder
    \item Sammenarbejdsform mellem gruppe og vejleder
    \begin{enumerate}
        \item Formelle sammenarbejdsformer (mail, kontaktpersoner, mødedato, osv.)
        \item Forventningsafklaring – hvad for vejledning er der behov for/mulighed for at få
    \end{enumerate}
    \item Præsentation af projekt
    \begin{enumerate}
        \item Fastsat af projektet kan opfylde alle faglige mål 
    \end{enumerate}
    \item Snak om iterativ arbejdsmetode
    \item Opbygning og indhold af problemformulering
    \item Fast ugentlig tid for vejledermøde
\end{enumerate}
 
 \subsection{Referat}
 Fremmødte: Tristan, Marcus, Martin, Mathias, Edward, Nikolaj 
 \\Udeblevet med afbud:
 \\Udeblevet uden afbud: Aaron, Martin F.
 \\Referent: Tristan 
 \\Dagsorden 
 \begin{enumerate}
    \item Ad) Præsentation af vejleder
    \begin{enumerate}
        \item Spørg om hjælp til process - men tungt teori, så spørg faglærer 
        \item Indtil eksamen er han vores mand. Til eksamen eksaminator.
    \end{enumerate}
     
    \item Ad) Samarbejde
    \begin{enumerate}
        \item Bedre indbyrdes kommunikation.
        \item Scrummaster skal styre kommunikation med vejleder og lave dagsorden til vejledermøde og gruppemøde
        \item Dagsorden skal sendes senest 24 timer inden møde (Weekend tæller ikke for vejleder) - Samt det skal indsættes i kalenderen og booke/finde et mødelokales
        \item Hjælper os med at uddybe vores kravsspecifikation så vi kan få det rigtige ambitionsniveau til eksamen.
        \item 2 uger før aflevering er der test fremlæggelse, hvor vi bliver klædt på til eksamen.
    \end{enumerate} 
    
    \item Ad) Præsentation af projekt
    \begin{enumerate}
        \item Processen er utrolig vigtigt, det er vigtigt at dokumenterer den iterative arbejdsprocess 
        \item Det "indlejret" er et vigtigt element i projektet - husk dette!
        \item GUI skal være der, men det er lige meget om det bare er to knapper. Avanceret GUI giver ikke plus point.
        \item De fleste kurser skal være dækket i forhold til teori - men ikke alt 
        \item Eventuel af måling af hvor mange bolde som er tilbage 
        \item Aktuater = "Påvirkning af omverdenen", fx højtaler (Der skal være noget "tyngde") 
        \item Vær sikker på at der er nok softwaredel 
        \item Gå i gang med must haves først - derefter begynd på should have osv.
    \end{enumerate}
    \\
    \item Ad) Iterativ arbejdsmetode
    \begin{enumerate}
        \item Beskriv eventuel tidsestimeringsprocessen
        \item Angiv hvor manger timer som skal bruges hvert sprint  
        \item Der kan godt være forskellige timer per person, hvis man kun vil give eks. antal timer. Her vil eksaminator kigge på eks antal timer lagt i projektet mv. Fx ved at kigge på vores trelloboard. 
        \item Det kan fx aftales internt, at hvis en ikke gider at ligge særligt mange timer ind i projektet, så kan dette aftales og scrum kan udarbejdes ud fra dette. 
        \item Vejleder kan bruges som konsulent i tilfælde af arbejdsproblemer i gruppen
        \item Tingene skal have et niveau så der kan arbejdes videre. 
    \end{enumerate}
    
    \item Ad) Opbygning og indhold af problemformulering
    \begin{enumerate}
        \item Fin struktur - kan eventuel specificeres længere inde i projektet 
        \item Går indirekte over i usecase, hvad skal brugeren gøre 
    \end{enumerate}
    
    \item Ad) Fast ugentlig tid for vejledermøde
    \begin{enumerate}
        \item Normalt : Mandag 12:15 til 13:15 
        \item Godkendt uger: 39, 40, 42, 43, 44, 45, 47, 49
        \item Ikke godkendt, uge 41:  Torsdag d. 11, 12:15 til 13:15 
        \item Ikke godkendt, uge 46:  (Skal tjekkes af vejleder) 
        \item Ikke godkendt, uge 48:  I stedet uge 47, Fredag d. 23, 8:30 
        \item Ekstra timer,  uge 50:  Onsdag d. 12, 12:00 til 16:00   
        \item Vejledermøde i januar planlægges i de sidste uger i december...
    \end{enumerate}

    
 \end{enumerate}

%________________________________________________________________________________________________________%
%________________________________________________________________________________________________________%
%________________________________________________________________________________________________________%
%___________________________________________NÆSTE VEJLEDERMØDE___________________________________________%
\newpage
\section{Vejledermøde den 24-09-2018}
\subsection{Mødeindkaldelse}
Dato: 24-09-2018
\\Tid: 12.15
\\Sted: Kælderen i Nygaard (Under Peter Bøgh)
\\Deltagere: Gruppe 7 og vejleder
\\Referent: 
\\Dagsorden:
\begin{enumerate}
    \item Use Cases
    \item Risiko Analyse
    \item Ikke Funktionelle Krav
    \item Aktør-kontekst Diagram
\end{enumerate}

\subsection{Referat}
 Fremmødte:
 \\Udeblevet med afbud: Martin G,
 \\Udeblevet uden afbud: 
 
 \begin{enumerate}
    \item 
\end{enumerate}


%________________________________________________________________________________________________________%
%________________________________________________________________________________________________________%
%________________________________________________________________________________________________________%
%_______________________________________________NÆSTE VEJLEDERMØDE_______________________________________%




\end{document}