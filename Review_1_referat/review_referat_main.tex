\documentclass[a4paper,12pt,fleqn,oneside]{article} 

\documentclass[a4paper,12pt,fleqn,oneside]{article} 	% Openright aabner kapitler paa hoejresider (openany begge)

%===== Projekt konstanter =====
\newcommand{\kursusTitel}{Semesterprojekt 3}
\newcommand{\linje}{E/IKT}
\newcommand{\semester}{3}
\newcommand{\system}{Beer Pong Master\copyright}
\newcommand{\rapportType}{Proces}
\newcommand{\gruppeNr}{7}
\newcommand{\vejleder}{Martin Ansbjerg Kjær}
\newcommand{\afleveringsdato}{6/6-2018}

%%%% PAKKER %%%%
% ¤¤ Oversaettelse og tegnsaetning ¤¤ %
\usepackage[utf8]{inputenc}					% Input-indkodning af tegnsaet (UTF8)
\usepackage[english, danish]{babel}					% Dokumentets sprog
\usepackage[T1]{fontenc}					% Output-indkodning af tegnsaet (T1)
\usepackage{ragged2e,anyfontsize}			% Justering af elementer

% ¤¤ Figurer og tabeller (floats) ¤¤ %
\usepackage{wrapfig} % text wrapping
\usepackage{graphicx} 						% Haandtering af eksterne billeder (JPG, PNG, PDF)
\usepackage{caption}
\usepackage{subcaption}
\usepackage{multirow}                		% Fletning af raekker og kolonner (\multicolumn og \multirow)
\usepackage{makecell}                       % Line breaks i tabelceller med \makecell{bla bla \\ bla bla}
\usepackage{colortbl} 						% Farver i tabeller (fx \columncolor, \rowcolor og \cellcolor)
\usepackage[dvipsnames,table,longtable,x11names]{xcolor}				% Definer farver med \definecolor. Se mere: http://en.wikibooks.org/wiki/LaTeX/Colors
\usepackage{flafter}						% Soerger for at floats ikke optraeder i teksten foer deres reference
\let\newfloat\relax 						% Justering mellem float-pakken og memoir
\usepackage{float}							% Muliggoer eksakt placering af floats, f.eks.
\usepackage{afterpage}
%\usepackage{scrextend}                      % labeling lister
\usepackage{chngcntr} % kontroller nummerering af floats
\counterwithin{figure}{section} % sæt nummerering efter sektion
\counterwithin{table}{section} % sæt nummerering efter sektion

% ¤¤ Matematik mm. ¤¤
\usepackage{amsmath,amssymb,stmaryrd} 		% Avancerede matematik-udvidelser
\usepackage{mathtools}						% Andre matematik- og tegnudvidelser
\usepackage{textcomp}                 		% Symbol-udvidelser (f.eks. promille-tegn med \textperthousand )
\usepackage{siunitx}						% Flot og konsistent praesentation af tal og enheder med \si{enhed} og \SI{tal}{enhed}
\sisetup{output-decimal-marker = {,}}		% Opsaetning af \SI (DE for komma som decimalseparator)

% ¤¤ Misc. ¤¤ %
\usepackage{listings}						% Placer kildekode i dokumentet med \begin{lstlisting}...\end{lstlisting}
\usepackage{blindtext}
\usepackage{lipsum}							% Dummy text \lipsum[..]
\usepackage[shortlabels]{enumitem}			% Muliggoer enkelt konfiguration af lister
\usepackage{pdfpages}						% Goer det muligt at inkludere pdf-dokumenter med kommandoen \includepdf[pages={x-y}]{fil.pdf}
\pdfoptionpdfminorversion=6					% Muliggoer inkludering af pdf dokumenter, af version 1.6 og hoejere
\pretolerance=2500 							% Justering af afstand mellem ord (hoejt tal, mindre orddeling og mere luft mellem ord)


%%%% BRUGERDEFINEREDE INDSTILLINGER %%%%

\linespread{1,1}							% Linie afstand

% ¤¤ Visuelle referencer ¤¤ %
\usepackage[colorlinks,pdfencoding=auto]{hyperref}			% Danner klikbare referencer (hyperlinks) i dokumentet.
\hypersetup{colorlinks = true,				% Opsaetning af farvede hyperlinks (interne links, citeringer og URL)
    linkcolor = black,
    citecolor = black,
    urlcolor = black
}

%%%% TODO-NOTER %%%%
\usepackage[danish, colorinlistoftodos]{todonotes}
%\usepackage[colorinlistoftodos]{todonotes}

%%%% TABEL BAGGRUNDSFARVER %%%%
\definecolor{aublueclassic}{RGB}{0,61,115}
\definecolor{aubluedark}{RGB}{0,37,70}
\definecolor{aucyan}{RGB}{225,248,253}
%\definecolor{aucyan}{RGB}{55,160,203}
\definecolor{aucyandark}{RGB}{0,62,92}
\definecolor{lightGray}{RGB}{153,153,153}
\definecolor{darkGray}{RGB}{119,119,119}
\definecolor{khaki}{RGB}{240,230,140}
\definecolor{lavender}{RGB}{230,230,250}

%Code highlighting
\usepackage{minted}

%%%% Tabs %%%%
\usepackage{tabto}
\NumTabs{10}

%%%% REFERENCE TIL SECTION-NAME %%%%
\usepackage{nameref}

% ========== PAKKER DER SKAL LOADES TIL SIDST ==================
%\usepackage{xcolor}
%\usepackage{listings}
\usepackage{csquotes}                       %så holder bilatex kæft

\def\titlename{Review 1 - referat}

\begin{document}

%===============FORSIDE======================
\input{Setup/Forside.tex}
\newpage

\section{Generelt}
\begin{itemize}
    \item Mangler versionshistorik
    \item Kravspec og accepttest skal måske adskilles
    \item Sørg for at have styr på dansk/engelsk
    \item God opdeling af Use Cases
\end{itemize}

\section{Projektformulering}
\begin{itemize}
    \item God skitse, mere af det
    \item Inkluder gerne en skitse af brugergrænseflade til at demonstrere UI
    \item Mere specifik på UI
\end{itemize}

\section{Kravspec}
\begin{itemize}
    \item Manglende tabelnumre
    \item Mangler Moscow
    \item Find officielle regler og vedhæft dem som bilag
    \item Vær mere konsistent og konkret med betegnelserne, nogle steder anvendes ordlisten, og andre steder gør den ikke
    \item Tegn tingene og bliv enig om betegnelserne hurtigst muligt
    \item God idé med definitionsliste, men lav den mere udtømmende
    \item Anvend listen i flowcharten på s. 5
    \item Aktør/kontekst: Mangler beskrivelse til diagrammet (hvem er worker?), aktiv bruger og modstander er to separate aktører, indsæt evt. sekundære aktører: kopper, bolde
    \item UC 1: Her vil det igen være godt med en beskrivelse, hvad er playerside, startup, angivne pladser, hvem henviser til hjemmeside, punkt 8 kunne godt være en postkondition, da den flyder sammen med næste UC. Eventuel prækondition: det antages at brugeren har kopper med, ellers kan der ikke spilles, I ext: husk at skriv, at UC afsluttes, hvad sker der med mønten her
    \item UC 2: Prækondition: definer 'angiv plads', reference til ordliste mangler, punkt 5: brug ordene 'tur skifter' som tidligere defineret, Kald det ext frem for exc, det er ikke sikkert, at der skal være en 'mulighed', den afsluttes brat
    \item UC 3: Første sætning giver ikke mening, brug udtrykket modstandersiden
    \item UC 4: Det er måske mere kompliceret end nødvendigt, god idé med LED, men hvad lyser den mellem 2 og 19 bolde, mål stemmer ikke overens med postkondition, kaldes det worker/servicemedarbejder, i hovedscenariet: hvilket UI henvises der til, definer låsen i ikke-funktionelle krav i stedet for i Use Casen, i ext 1: bolddispenseren lukkes 2 gange?, at bolddispenseren er fyldt op skal måske være en del af hovedscenariet
    \item Ikke-funktionelle krav: Typen af mønt/kop/bold mangler, er det metalkugle / bordtennisbold, her kunne også godt bruges en tegning, mangler tolerancer på mange af kravene, hvad er kravene til displayet, definer tingene i kursiv nærmere, ordet 'øjeblikkeligt' er ikke testbart, definer vandtæt, skal definere hvad brugeren må komme med af bolde og kopper
\end{itemize}

\section{Accepttest}
\begin{itemize}
    \item UC 1: Bruger indsætter mønt i bolddispenser - hvilken mønt, hvor udstedes boldene, hvilket lys i playerside tændes, skriv handlingen som en manual til en testperson frem for i 3. person, specificer hvilken tekst der står på displayet / hjemmesiden, 
    til extension: mangler en prækondition med ingen bolde i systemet
    \item UC 2: Hvis man ikke rammer, hvorfor skal der så fjernes en kop
    \item UC 3: Der skal beskrives hvad brugeren gør, frem for hvad systemet gør, her introduceres noget funktionalitet, som ikke er beskrevet før 
    \item UC 4: Stemmer ikke helt overens med UC, navnene er ikke konsekvente, mismatch i forhold til postkondition, igen mere specifik på UI
\end{itemize}

\section{Arkitektur}
\begin{itemize}
    \item Møntindkastet skal beskrives tidligere (at det er en del af samme enhed, ball dispenser) 
    \item Gode BDD'er, det første overordnede BDD kan godt udvides og efterfølges af systemsekvensdiagrammer (muligvis personlig præference)
    \item Domænemodellen kan med fordel flyttes op, da den giver et overblik over systemet 
    \item BDD: Måske kopholdere frem for kopper
    \item IBD på s. 5: Skriv gerne hvorfor cups, coins, balls ikke er med
    \item Ball dispenser på s 7: Det er først her det angives, at det er 5kr
    \item s 8: Hold diagrammer på ét niveau
    \item Sekvensdiagrammer: Navne på UC stemmer ikke overens - dansk/engelsk, UC2 er billedet for UC4, nogle steder kan blokkene skubbes mere sammen, så det fylder mindre
    \item Grænseflader: Fjern navnet Beerpong Master 2000, protokol: der er ingen grund til at beskrive hvordan I2C virker, beskriv hellere hvordan vi bruger det
    \item s. 17: Her introduceres begrebet game history for første gang
    \item Domænemodel: Den giver overblik over systemet, så den kunne med fordel inkluderes før BDD osv.
    \item s. 19: I2C protokol skal skrives som domæne nede i teksten
    \item Klassediagram for RPi: Det er uoverskueligt, vis eventuelt det første klassediagram uden metoder og vis så 1 UC af gangen
    \item Sekvensdiagrammerne: Mange af kaldene der laves er de samme metoder. Der kunne godt laves flere separate metoder, med mere specifikke navne (så læser bedre kan se, hvad der foregår)
    \item s. 26: Mangler entry points idle på nogle af state machines
    \item s. 28: Klassediagrammet følger ikke helt UML standarden med returtype
    \item Sekvensdiagram på s. 30: Måske et bedre navn til dispenseoption
    \item God tekst på side 33 sammen med tilstandsmaskinen
\end{itemize}

\end{document}