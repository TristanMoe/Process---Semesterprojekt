\documentclass{article}
\documentclass[a4paper,12pt,fleqn,oneside]{article} 	% Openright aabner kapitler paa hoejresider (openany begge)

%===== Projekt konstanter =====
\newcommand{\kursusTitel}{Semesterprojekt 3}
\newcommand{\linje}{E/IKT}
\newcommand{\semester}{3}
\newcommand{\system}{Beer Pong Master\copyright}
\newcommand{\rapportType}{Proces}
\newcommand{\gruppeNr}{7}
\newcommand{\vejleder}{Martin Ansbjerg Kjær}
\newcommand{\afleveringsdato}{6/6-2018}

%%%% PAKKER %%%%
% ¤¤ Oversaettelse og tegnsaetning ¤¤ %
\usepackage[utf8]{inputenc}					% Input-indkodning af tegnsaet (UTF8)
\usepackage[english, danish]{babel}					% Dokumentets sprog
\usepackage[T1]{fontenc}					% Output-indkodning af tegnsaet (T1)
\usepackage{ragged2e,anyfontsize}			% Justering af elementer

% ¤¤ Figurer og tabeller (floats) ¤¤ %
\usepackage{wrapfig} % text wrapping
\usepackage{graphicx} 						% Haandtering af eksterne billeder (JPG, PNG, PDF)
\usepackage{caption}
\usepackage{subcaption}
\usepackage{multirow}                		% Fletning af raekker og kolonner (\multicolumn og \multirow)
\usepackage{makecell}                       % Line breaks i tabelceller med \makecell{bla bla \\ bla bla}
\usepackage{colortbl} 						% Farver i tabeller (fx \columncolor, \rowcolor og \cellcolor)
\usepackage[dvipsnames,table,longtable,x11names]{xcolor}				% Definer farver med \definecolor. Se mere: http://en.wikibooks.org/wiki/LaTeX/Colors
\usepackage{flafter}						% Soerger for at floats ikke optraeder i teksten foer deres reference
\let\newfloat\relax 						% Justering mellem float-pakken og memoir
\usepackage{float}							% Muliggoer eksakt placering af floats, f.eks.
\usepackage{afterpage}
%\usepackage{scrextend}                      % labeling lister
\usepackage{chngcntr} % kontroller nummerering af floats
\counterwithin{figure}{section} % sæt nummerering efter sektion
\counterwithin{table}{section} % sæt nummerering efter sektion

% ¤¤ Matematik mm. ¤¤
\usepackage{amsmath,amssymb,stmaryrd} 		% Avancerede matematik-udvidelser
\usepackage{mathtools}						% Andre matematik- og tegnudvidelser
\usepackage{textcomp}                 		% Symbol-udvidelser (f.eks. promille-tegn med \textperthousand )
\usepackage{siunitx}						% Flot og konsistent praesentation af tal og enheder med \si{enhed} og \SI{tal}{enhed}
\sisetup{output-decimal-marker = {,}}		% Opsaetning af \SI (DE for komma som decimalseparator)

% ¤¤ Misc. ¤¤ %
\usepackage{listings}						% Placer kildekode i dokumentet med \begin{lstlisting}...\end{lstlisting}
\usepackage{blindtext}
\usepackage{lipsum}							% Dummy text \lipsum[..]
\usepackage[shortlabels]{enumitem}			% Muliggoer enkelt konfiguration af lister
\usepackage{pdfpages}						% Goer det muligt at inkludere pdf-dokumenter med kommandoen \includepdf[pages={x-y}]{fil.pdf}
\pdfoptionpdfminorversion=6					% Muliggoer inkludering af pdf dokumenter, af version 1.6 og hoejere
\pretolerance=2500 							% Justering af afstand mellem ord (hoejt tal, mindre orddeling og mere luft mellem ord)


%%%% BRUGERDEFINEREDE INDSTILLINGER %%%%

\linespread{1,1}							% Linie afstand

% ¤¤ Visuelle referencer ¤¤ %
\usepackage[colorlinks,pdfencoding=auto]{hyperref}			% Danner klikbare referencer (hyperlinks) i dokumentet.
\hypersetup{colorlinks = true,				% Opsaetning af farvede hyperlinks (interne links, citeringer og URL)
    linkcolor = black,
    citecolor = black,
    urlcolor = black
}

%%%% TODO-NOTER %%%%
\usepackage[danish, colorinlistoftodos]{todonotes}
%\usepackage[colorinlistoftodos]{todonotes}

%%%% TABEL BAGGRUNDSFARVER %%%%
\definecolor{aublueclassic}{RGB}{0,61,115}
\definecolor{aubluedark}{RGB}{0,37,70}
\definecolor{aucyan}{RGB}{225,248,253}
%\definecolor{aucyan}{RGB}{55,160,203}
\definecolor{aucyandark}{RGB}{0,62,92}
\definecolor{lightGray}{RGB}{153,153,153}
\definecolor{darkGray}{RGB}{119,119,119}
\definecolor{khaki}{RGB}{240,230,140}
\definecolor{lavender}{RGB}{230,230,250}

%Code highlighting
\usepackage{minted}

%%%% Tabs %%%%
\usepackage{tabto}
\NumTabs{10}

%%%% REFERENCE TIL SECTION-NAME %%%%
\usepackage{nameref}

% ========== PAKKER DER SKAL LOADES TIL SIDST ==================
%\usepackage{xcolor}
%\usepackage{listings}
\usepackage{csquotes}                       %så holder bilatex kæft
%\usepackage{biblatex}
%\addbibresource{ref.bib}

\begin{document}

\def\titlename{Risiko analyse}
\input{Setup/Forside.tex}
\section{Overordnede risikoer}
%\input{Risiko analyse/overordnede.tex}


\begin{table}[H]
\resizebox{\textwidth}{!}{%
\begin{tabular}{|p{5cm}|p{1cm}|p{1.2cm}|p{1.2cm}|p{5cm}|}
\hline
\textbf{Description} & \textbf{prob. 1-5} & \textbf{conseq. 1-5} & \textbf{impact 1-25} & \textbf{plan} \\ \hline
Presset med andre afleveringer & 4 & 5 & 20 & Vigtigt at fortælle om man har tid til en opgave, så gruppen kan planlægge ting ud fra det. \\ \hline
Vi er ikke klar over kompleksitet af de ting vi prøver at lave. & 3 & 4 & 12 & Spørger vejleder eller andre lærere om de mener det er muligt med den viden vi har.\\ \hline
Sygdom & 4 & 2 & 8 & Andre tager over med opgaver. \\ \hline
En person dropper ud & 2 & 5 & 10 & Ikke have for mange ting i moscow  must have, så vi kan tåle at ting, som dette sker. \\ \hline
Vi er ikke nok elektronik folk & 2 & 1 & 2 & Nogle ikt’er hjælper til med elektronikken. \\ \hline
Få fat i ting til at lave beer pong bordet. & 2 & 4 & 8 & Vi må alle gøre en indsats, så der ikke er få der hænger på den. \\ \hline
Teknisk uforudsete ting & 2 & 4 & 8 & Der må laves et grundigt design og analyse, så dette ikke sker. \\ \hline
Manglende arbejdsindsats fra personer & 2 & 5 & 10 & Manglende arbejdsindsats fra personer. \\ \hline
Folk er bagud og har ikke nok teknisk forståelse. & 2 & 5 & 10 & Folk er bagud og har ikke nok teknisk forståelse. \\ \hline
Folk er ikke villige til at arbejde over hvis det bliver nødvendigt. & 1 & 2 & 2 & Der planlægges bedre og folk siger det i forvejen for det bliver travlt på et tidspunkt. \\ \hline
Folk spørger ikke om hjælp hvis de har brug for det & 2 & 4 & 8 & Folk er gode til at hjælpe og spørger om folk har brug for det. \\ \hline

\end{tabular}%
}
\end{table}

\section{Tekniske risikoer}

\begin{table}[H]
\resizebox{\textwidth}{!}{%
\begin{tabular}{|p{5cm}|p{1cm}|p{1.2cm}|p{1.2cm}|p{5cm}|}
\hline
\textbf{Description} & \textbf{prob. 1-5} & \textbf{conseq. 1-5} & \textbf{impact 1-25} & \textbf{plan} \\ \hline
Sensorer,til at detekterer flytning af kop, bliver ikke lavet eller der opstår problemer. & 2 & 5 & 10 & Det ser ikke ud til dette bliver et problem at få denne del lavet, men da det er meget vigtigt for spillet derfor har den fået conseq 5.  \\ \hline
Sensorer til Dispenser & 3 & 4 & 12 & Vigtig fordi den er i vores must have, men spillet ville kunne fungerer uden. Den burde ikke være vildt svær at lave. Den skal med i sprints fra starten. \\ \hline
Aktuator/motor til dispensering. & 3 & 4 & 12 & Skal med i sprints fra starten.  \\ \hline
Detektering af mønter. & 3 & 4 & 12 & Få det til at virke med alle mønter først og herefter til specifikke mønter. \\ \hline
Belysning til under/rundt om hver kop. & 1 & 4 & 4 & Rimeligt vigtigt i forhold til det at flytte koppen, da det er en måde at se det er gjort. I forhold til at skulle lave dette på en simpel måde er det ikke det største problem. Her er det vigtigt at vi fokusere på de første 6 kopholdere og derefter udvider, hvis der er tid. \\ \hline
UI til visning af scoreboard osv. & 4 & 5 & 20 & Der kan gøres meget med dette og da mange funktionaliteter påvirker funktionaliteten af UI, så kunne der godt opstå mange problemer her. Evt kan flere personer have fokus på dette område og det er vigtigt at folk der har med fx flytning af kop har UI i tankerne og er enige med UI folket om grænseflader. \\ \hline
Registrering af om kop er ramt. & 5 & 2 & 10 & En ‘should have’ i projektet. Der vendtes til de andre vigtige ting er med, før der fokuseres på dette(ikke med i de første sprints). \\ \hline
Timer efter flytning af cup. & 1 & 1 & 1 & Venter til aller sidst med dette hvis der er tid. \\ \hline
Indtastning af information på Website & 3 & 3 & 9 & Impact afhænger af hvad vi vil gøre ud af det kan laves forholdsvis simpelt. Plannen starte ud simpelt og udvide hvis der bliver tid, men skal ikke på de første sprints.Kan evt. Også droppes, da den er ‘should have’. Should have prioritet. \\ \hline
12 kopholdere istedet for 6 & 1 & 2 & 2 & Der laves de første 6 kopholdere til at begynde med, men da det er sjovt at have alle 12 til at kunne få et ordentligt spil ud af det, har den høj prioritet af tingene i ‘should have’. \\ \hline
Hjemmeside hvor resultater gemmes & 3 & 5 & 15 & Har skiftet prioritet til at være høj, da vi har besluttet at det er en must have. Det er over denne hjemmeside man evt. bruger mobil til at indtaste informationer og uden hjemmesiden kan vi ikke indtaste noget information(fx spiller navne) \\ \hline
Belysning rundt om bord & 1 & 1 & 1 & Kan måske komme på tale. \\ \hline

\end{tabular}%
}
\end{table}

\section{Faglige mål}

\begin{table}[H]
\resizebox{\textwidth}{!}{%
\begin{tabular}{|p{5cm}|p{1cm}|p{1.2cm}|p{1.2cm}|p{5cm}|}
\hline
\textbf{Description} & \textbf{prob. 1-5} & \textbf{conseq. 1-5} & \textbf{impact 1-25} & \textbf{plan} \\ \hline
Der er ikke noget fra faget DOA med. & 1 & 4 & 4 & Have fokus på dette i implementeringen af alt det kode der skrives om der ikke kan inkluderes lidt her. \\ \hline
Der er noget fra digital signal behandling med. & 2 & 2 & 4 & Hvis det er det eneste fag vi ikke har noget med fra. \\ \hline
Der er nok fra E’ernes timer med i det de laver. & & & & \\ \hline
\end{tabular}%
}
\end{table}

\section{Praktiske risikoer}

\begin{table}[H]
\resizebox{\textwidth}{!}{%
\begin{tabular}{|p{5cm}|p{1cm}|p{1.2cm}|p{1.2cm}|p{5cm}|}
\hline
\textbf{Description} & \textbf{prob. 1-5} & \textbf{conseq. 1-5} & \textbf{impact 1-25} & \textbf{plan} \\ \hline
Et rør til dispenser. & & & & \\ \hline
Hvor skal mønterne være? & & & & \\ \hline
En motor til dispenser. Kan vi låne en steppermotor? & & & & \\ \hline
En glas plade til at have over sensorer. & & & & \\ \hline
Finde en skærm vi kan bruge & & & & \\ \hline
Beskyttelse af skærm imod øl & & & & \\ \hline

\end{tabular}%
}
\end{table}

\end{document}